\documentclass{article}
\usepackage{graphicx} % Required for inserting images

\title{Mozila dataset project ideas}
\author{ Naser Ezzati-Jivan }
\date{December 2025}

\begin{document}
\maketitle
\section{Phase 2: Multi-Class Alert Status Prediction}
\label{sec:phase2}



\subsection{Goal and Motivation}
Phase 2 extends the binary regression classification problem introduced in Phase 1 into a multi-class prediction task. Each alert in Mozilla’s dataset is assigned a status that reflects the decision made by performance engineers during triage. The possible statuses include \texttt{Invalid}, \texttt{Downstream}, \texttt{Reassigned}, \texttt{Improvement}, \texttt{Investigating}, \texttt{Wontfix}, \texttt{Fixed}, and \texttt{Backedout}. These categories represent different engineering actions and reasoning steps, and predicting them can help quantify the triage workflow and identify factors that influence engineering decisions.

From a research perspective, Phase 2 asks whether metadata alone can capture the complex decision-making process of human engineers. The student learns how to manage imbalanced data, evaluate multi-class models, and interpret results using per-class metrics. This phase builds on the foundation of Phase 1 and contributes toward understanding performance triage in large-scale continuous integration systems.

\subsection{Objectives and Research Questions}

The main objective of Phase 2 is to train and evaluate a multi-class classification model that predicts the \texttt{single\_alert\_status} for each alert. This provides insight into patterns in the triage process and prepares the foundation for more complex tasks in later phases.

\subsubsection{Main Objective}
The primary objective is to determine whether the alert-level metadata that was used in Phase 1 can also predict specific triage outcomes assigned by Mozilla engineers, rather than only the binary regression label.

\subsubsection{Sub-Objectives}
\begin{itemize}
    \item Identify which metadata fields are most informative for predicting triage outcomes.
    \item Handle imbalanced classes using techniques such as class weighting or synthetic oversampling.
    \item Evaluate model performance using per-class precision, recall, F1-score, and confusion matrices.
    \item Compare simple models (Logistic Regression with softmax) with tree-based models (Random Forest, Gradient Boosting).
    \item Test generalization by training on alerts from certain repositories and evaluating on unseen repositories.
    \item Perform ablation studies to understand the importance of magnitude features, contextual metadata, and workflow-related fields.
\end{itemize}

\subsubsection{Research Questions}
\begin{itemize}
    \item \textbf{RQ1:} Can alert metadata accurately predict the human-assigned status of an alert in Mozilla's performance triage workflow?
    \item \textbf{RQ2:} Which metadata features are most strongly associated with specific statuses such as \texttt{Invalid}, \texttt{Downstream}, and \texttt{Fixed}?
    \item \textbf{RQ3:} Are there significant differences in model accuracy across repositories, platforms, or test suites?
    \item \textbf{RQ4:} How do magnitude-based features compare to contextual features when predicting complex triage decisions?
    \item \textbf{RQ5:} Can a multi-class model highlight ambiguous or inconsistent triage decisions, suggesting areas where automation might assist engineers?
\end{itemize}

\subsection{Data Used}
Phase 2 uses the same file as Phase 1:
\begin{itemize}
    \item \textbf{\texttt{data/alerts\_data.csv}}: contains metadata for all alerts, including the status assigned during triage.
\end{itemize}

The target variable is:
\begin{itemize}
    \item \textbf{\texttt{single\_alert\_status}}: the triage outcome assigned by Mozilla engineers.
\end{itemize}

Unlike Phase 1, the label has multiple categories, and some classes may be rare.

\subsection{Problem Definition}
We define a multi-class classification problem:
\begin{equation}
    f(\mathbf{x}) \rightarrow y,\ \ \ y \in \{\text{Invalid}, \text{Downstream}, \text{Fixed}, \text{Wontfix}, \text{Improvement}, ...\}
\end{equation}
where $\mathbf{x}$ is the metadata feature vector, and $y$ is the triage status.

\subsection{Feature Set}
The feature set for Phase 2 is identical to Phase 1, but workflow features become more relevant because the goal is to model triage behavior, not only detect regressions.

\paragraph{Magnitude and statistical features.}
\begin{itemize}
    \item \texttt{single\_alert\_amount\_abs}
    \item \texttt{single\_alert\_amount\_pct}
    \item \texttt{single\_alert\_t\_value}
\end{itemize}

\paragraph{Contextual features.}
\begin{itemize}
    \item \texttt{repository\_name}
    \item \texttt{machine\_platform}
    \item \texttt{framework\_id}
    \item \texttt{single\_alert\_series\_signature\_suite}
\end{itemize}

\paragraph{Workflow features.}
These become more important in Phase 2:
\begin{itemize}
    \item \texttt{single\_alert\_manually\_created}
    \item \texttt{alert\_summary\_assignee\_username}
    \item \texttt{alert\_summary\_bug\_number} (include only if set at creation time)
\end{itemize}

Leakage checks from Phase 1 still apply: no features from after triage should be included.

\subsection{Implementation Steps}

\paragraph{Step 1: Data loading and preprocessing.}
Load \texttt{alerts\_data.csv}, filter out alerts with missing \texttt{single\_alert\_status}, and drop identifiers.

\paragraph{Step 2: Handle class imbalance.}
\begin{itemize}
    \item Compute class counts.
    \item Apply class weights in Logistic Regression or tree models.
    \item Optionally use SMOTE for minority classes.
\end{itemize}

Example:
\begin{verbatim}
from sklearn.utils.class_weight import compute_class_weight
classes = alerts["single_alert_status"].unique()
weights = compute_class_weight("balanced", classes=classes,
                               y=alerts["single_alert_status"])
\end{verbatim}

\paragraph{Step 3: Encode categorical variables.}
Use one-hot encoding for small categories and frequency encoding for large categories.

\paragraph{Step 4: Model selection.}
Models suitable for multi-class classification:
\begin{itemize}
    \item Logistic Regression with softmax
    \item Random Forest
    \item XGBoost or LightGBM with multi-class objective
\end{itemize}

\paragraph{Step 5: Evaluation.}
Compute:
\begin{itemize}
    \item Per-class precision and recall
    \item Macro- and micro-averaged F1-score
    \item Full confusion matrix
\end{itemize}

\paragraph{Step 6: Feature importance and interpretation.}
Use:
\begin{itemize}
    \item Tree-based feature importance
    \item Permutation importance
    \item SHAP values for multi-class outputs
\end{itemize}

\subsection{Experimental Design}

\paragraph{Experiment E1: Baseline multi-class classification.}
Train Logistic Regression, Random Forest, and XGBoost on the full feature set. Evaluate which model handles imbalance and multi-class structure best.

\paragraph{Experiment E2: Repository-level generalization.}
Train on autoland repositories and test on mozilla-central or mozilla-beta to determine whether triage patterns differ across contexts.

\paragraph{Experiment E3: Feature ablation for triage prediction.}
Evaluate the accuracy of:
\begin{itemize}
    \item magnitude-only features,
    \item context-only features,
    \item workflow-only features,
    \item combinations of these groups.
\end{itemize}

This clarifies which aspects of alert metadata influence triage outcomes.

\subsection{Concrete Examples of Alerts}

\paragraph{Example 1: Invalid alert.}
A low-magnitude alert occurring on a noisy platform is often labeled as \texttt{Invalid}. These alerts form a large portion of the dataset.

\paragraph{Example 2: Downstream alert.}
Downstream alerts appear shortly after a primary regression and often share the same summary. Metadata about the summary or test suite can help identify them.

\paragraph{Example 3: Fixed alert.}
If a corresponding bug is filed and resolved promptly, the alert may ultimately receive the status \texttt{Fixed}. Early metadata may partially reflect this pattern.

\subsection{Expected Outcomes}
Phase 2 is expected to produce:
\begin{itemize}
    \item A multi-class classifier that models Mozilla's triage decisions.
    \item A confusion matrix showing which statuses are hard to predict.
    \item A ranked list of features affecting specific triage outcomes.
    \item Insights into the consistency or inconsistency of triage behavior.
\end{itemize}

\subsection{Reproducibility and Reporting}
The student will maintain:
\begin{itemize}
    \item preprocessing scripts,
    \item model training and evaluation notebooks,
    \item per-class plots and confusion matrices,
    \item a written report summarizing findings.
\end{itemize}

\subsection{Risks and Mitigations}
\begin{itemize}
    \item \textbf{Extreme class imbalance:} mitigate through class weighting and resampling.
    \item \textbf{Label ambiguity:} some statuses may overlap conceptually; use coarse-grained grouping if needed.
    \item \textbf{Feature leakage:} verify timestamps to ensure workflow features do not include post-triage information.
\end{itemize}

\subsection{Phase 2 Deliverables}
At the end of Phase 2 the student will deliver:
\begin{itemize}
    \item A fully implemented multi-class classifier.
    \item Model evaluations with per-class breakdown.
    \item A generalization study across repositories.
    \item A short written summary connecting findings back to the goals of regression triage analysis.
\end{itemize}

\end{document}
