\documentclass{article}
\usepackage{graphicx} % Required for inserting images

\title{Mozila dataset project (ansh and deep patel)}
\author{ Naser Ezzati-Jivan }

\date{December 2025}

\begin{document}

\maketitle

\section{Integration Phase: Cross-Phase Synthesis and System-Level Evaluation}
\label{sec:integration}

\subsection{Goal and Motivation}
The Integration Phase combines the results and methodologies from all previous phases into a unified evaluation and analysis framework. Whereas Phases 1–6 focused on individual modeling tasks—metadata classification, multi-class prediction, time-series feature engineering, change-point detection, forecasting, and root cause analysis—the goal of this phase is to synthesize these components and assess how well they work together as a coherent system.

This phase evaluates the effectiveness of each approach, identifies complementary strengths, and explores whether combining multiple signals (metadata, time-series features, forecast residuals, and RCA outputs) can produce a more accurate and robust regression detection and triage framework. This integrated perspective reflects how real-world engineering workflows operate and helps determine the potential for automation in performance triage.

\subsection{Objectives and Research Questions}

\subsubsection{Main Objective}
The primary objective of the Integration Phase is to combine outputs from all phases to create an end-to-end regression detection and analysis pipeline, then evaluate its accuracy, generalization, interpretability, and potential usefulness for Mozilla’s performance engineering workflow.

\subsubsection{Sub-Objectives}
\begin{itemize}
    \item Integrate metadata-driven classification (Phases 1–2) with time-series features (Phase 3), change-point outputs (Phase 4), and forecast residuals (Phase 5).
    \item Develop a combined model or ensemble to predict regressions using all available feature types.
    \item Connect regression predictions with automated RCA categories (Phase 6).
    \item Evaluate end-to-end performance across repositories, platforms, and test suites.
    \item Identify inconsistencies, complementary signals, and cases where certain models outperform others.
    \item Produce a unified visualization and reporting dashboard demonstrating the full analysis pipeline.
\end{itemize}

\subsubsection{Research Questions}
\begin{itemize}
    \item \textbf{RQ1:} Does combining metadata, time-series features, change-point detections, and forecast residuals improve regression detection accuracy?
    \item \textbf{RQ2:} Which inputs contribute the most to an integrated model, and how do they interact?
    \item \textbf{RQ3:} Does an end-to-end model improve generalization across repositories compared to individual models?
    \item \textbf{RQ4:} Can integrated RCA outputs help prioritize alerts or identify likely root causes?
    \item \textbf{RQ5:} What are the limitations and practical barriers to deploying an integrated regression analysis system in CI environments?
\end{itemize}

\subsection{Data and Inputs}
This phase uses all outputs generated in earlier phases:
\begin{itemize}
    \item Metadata features (Phases 1–2),
    \item Time-series engineered features (Phase 3),
    \item Change-point detection outputs (Phase 4),
    \item Forecasting residuals and prediction intervals (Phase 5),
    \item RCA clusters, bug predictions, and extracted topics (Phase 6).
\end{itemize}

Each alert is represented by an extended feature vector combining all these elements.

\subsection{Integrated System Architecture}

The student constructs an integrated pipeline with the following components:
\begin{enumerate}
    \item \textbf{Input Preparation:} Merge metadata, time-series features, change-point signals, and forecasting residuals into a unified table.
    \item \textbf{Feature Normalization:} Scale numeric features and encode categorical RCA outputs where necessary.
    \item \textbf{Ensemble Model:} Use a meta-classifier (for example Gradient Boosting or a stacking ensemble) to combine signals.
    \item \textbf{RCA Linking:} Attach predicted root-cause labels or cluster identifiers to each detected regression.
    \item \textbf{Reporting Layer:} Generate summaries, dashboards, and visualizations showing detected regressions and predicted causes.
\end{enumerate}

\subsection{Implementation Steps}

\paragraph{Step 1: Aggregate all feature sources.}
Merge outputs from Phases 1–6 into a single DataFrame indexed by alert ID.

\paragraph{Step 2: Feature consistency and cleanup.}
Ensure that all feature groups follow consistent naming, scaling, and missing-value handling.

\paragraph{Step 3: Ensemble model construction.}
Implement stacking or blending strategies:
\begin{itemize}
    \item Level-1 models: metadata classifier, time-series classifier, change-point detector, forecasting anomaly detector.
    \item Level-2 meta-classifier: final regression predictor using all outputs as inputs.
\end{itemize}

\paragraph{Step 4: RCA integration.}
Attach RCA cluster labels, predicted bug existence, or component categories to each regression.

\paragraph{Step 5: End-to-end evaluation.}
Evaluate:
\begin{itemize}
    \item overall accuracy of regression detection,
    \item precision for real regressions,
    \item reduction in false positives,
    \item usefulness of RCA labels for triage.
\end{itemize}

\paragraph{Step 6: Visualization and reporting.}
Produce:
\begin{itemize}
    \item model comparison plots,
    \item confusion matrices,
    \item regression timelines,
    \item change-point overlays,
    \item annotated RCA summaries.
\end{itemize}

\subsection{Experimental Design}

\paragraph{Experiment E1: Integrated vs.\ individual model comparison.}
Compare ensemble performance with baseline models from Phases 1–5.

\paragraph{Experiment E2: Cross-suite integrated evaluation.}
Evaluate the integrated model across frameworks and platforms.

\paragraph{Experiment E3: Ablation of integrated components.}
Test:
\begin{itemize}
    \item metadata-only,
    \item metadata + time-series,
    \item metadata + time-series + forecasting,
    \item full integration with RCA.
\end{itemize}

\paragraph{Experiment E4: RCA usefulness analysis.}
Check whether alerts within a predicted RCA cluster correspond to shared bug IDs or components.

\subsection{Concrete Examples}

\paragraph{Example 1: Combined signals confirm a regression.}
A true regression exhibits:
\begin{itemize}
    \item metadata-based prediction = 1,
    \item slope and variance changes from Phase 3,
    \item change points detected near the regression index,
    \item forecast residuals exceeding prediction intervals,
    \item RCA cluster linking related alerts.
\end{itemize}

\paragraph{Example 2: Noisy false positive filtered out.}
An alert may have high magnitude but:
\begin{itemize}
    \item no clear time-series slope change,
    \item prediction intervals not violated,
    \item no cluster of related alerts,
    \item no consistent RCA signals.
\end{itemize}
The integrated model should classify it as noise.

\subsection{Expected Outcomes}
\begin{itemize}
    \item A unified regression detection and RCA pipeline.
    \item Improved accuracy over individual models due to complementary signals.
    \item A structured understanding of which components matter most.
    \item A demonstration of how automated alerts can support engineer triage.
\end{itemize}

\subsection{Reproducibility and Reporting}
The student will deliver:
\begin{itemize}
    \item unified datasets used for integration,
    \item model training scripts and configuration files,
    \item end-to-end pipeline notebooks,
    \item integrated performance reports and visual dashboards.
\end{itemize}

\subsection{Risks and Mitigations}
\begin{itemize}
    \item \textbf{Feature explosion:} apply PCA or feature selection before ensemble modeling.
    \item \textbf{Incompatible scales or units:} normalize and standardize prior to integration.
    \item \textbf{Model overfitting:} use repository-level cross-validation.
    \item \textbf{Sparse RCA signals:} rely more heavily on metadata and time-series features when text data is insufficient.
\end{itemize}

\subsection{Integration Phase Deliverables}
\begin{itemize}
    \item A full ensemble model combining metadata, time-series, change-point, forecasting, and RCA signals.
    \item Evaluation of cross-phase synergies and weaknesses.
    \item A complete workflow diagram for Mozilla-style regression analysis.
    \item A final comprehensive report summarizing all phases and providing recommendations for future work or deployment.
\end{itemize}



\end{document}
