\documentclass{article}
\usepackage{graphicx} % Required for inserting images

\title{Mozila dataset project (ansh and deep patel)}
\author{ Naser Ezzati-Jivan }

\date{December 2025}

\begin{document}

\maketitle
\section{Phase 6: Automated Root Cause Analysis}
\label{sec:phase6}

\subsection{Goal and Motivation}
Phase 6 integrates all results from the previous phases to explore automated root cause analysis (RCA). Root cause analysis is a central task in Mozilla’s performance triage process, where engineers analyze regressions, consult bug reports, and identify the subsystem, test, or component responsible for a performance issue. Mozilla’s dataset provides metadata, limited textual descriptions, alert relations, and bug information that can be used to infer likely root causes or cluster related alerts.

This phase requires advanced methods and represents the most open-ended and research-driven component of the project. The goal is to explore whether automated techniques can aid or partially automate triage by identifying patterns that indicate why a regression occurred.

\subsection{Objectives and Research Questions}

\subsubsection{Main Objective}
The main objective of Phase~6 is to design and evaluate methods that infer, group, or predict potential root causes using alert metadata, time-series signals, and bug report information.

\subsubsection{Sub-Objectives}
\begin{itemize}
    \item Cluster alerts based on combined metadata and time-series feature vectors to identify groups of related regressions.
    \item Predict which alerts are likely to result in bug reports using supervised learning.
    \item Apply text mining or simple NLP on alert notes and bug summaries to extract root-cause categories.
    \item Investigate whether alert linkages (for example downstream alerts) help identify primary regressions.
    \item Evaluate how well automated methods match or approximate Mozilla engineer decisions.
\end{itemize}

\subsubsection{Research Questions}
\begin{itemize}
    \item \textbf{RQ1:} Can we automatically group alerts that share the same root cause based on metadata and time-series behavior?
    \item \textbf{RQ2:} Which features (time-series variability, alert status, workflow features) help predict whether an alert will lead to a bug report?
    \item \textbf{RQ3:} Can textual information in bug summaries or alert notes reveal categories of regressions?
    \item \textbf{RQ4:} Do downstream alerts form coherent clusters that identify primary root causes?
    \item \textbf{RQ5:} How accurate are automated RCA predictions compared with Mozilla’s actual triage records?
\end{itemize}

\subsection{Data Used}
Phase 6 integrates all previously used data sources:
\begin{itemize}
    \item \textbf{\texttt{alerts\_data.csv}}: contains statuses, assignees, summary links.
    \item \textbf{\texttt{timeseries\_data/\*/\*.csv}}: provides performance behavior.
    \item \textbf{\texttt{bugs\_data.csv}}: includes bug summaries, components, creators, timestamps.
\end{itemize}

\subsection{Possible RCA Tasks}

\paragraph{Task 1: Alert Clustering.}
Use clustering algorithms (K-Means, DBSCAN, HDBSCAN) on engineered features to identify regression clusters. Compare clusters with shared bug IDs, platforms, or suites.

\paragraph{Task 2: Bug Prediction.}
Define a binary label:
\[
y = \begin{cases}
1 & \text{if alert leads to a bug report} \\
0 & \text{otherwise}
\end{cases}
\]
Train classifiers (Gradient Boosting, Random Forest) to predict which alerts require filing a bug.

\paragraph{Task 3: Root-Cause Category Extraction.}
Use text-based methods:
\begin{itemize}
    \item TF–IDF feature extraction,
    \item clustering on summary texts,
    \item keyword extraction (for example ``memory'', ``graphics'', ``I/O'', ``network'').
\end{itemize}

\paragraph{Task 4: Downstream Analysis.}
Downstream alerts may share root causes with the primary alert. Evaluate whether graph-based grouping of alerts within the same summary leads to coherent clusters.

\subsection{Implementation Steps}

\paragraph{Step 1: Feature preparation.}
Combine metadata and time-series features from Phases 1–3 into a unified vector.

\paragraph{Step 2: Construct RCA labels.}
Construct labels such as:
\begin{itemize}
    \item bug existence,
    \item bug component,
    \item alert summary ID,
    \item downstream versus primary alert.
\end{itemize}

\paragraph{Step 3: Perform clustering.}
Run clustering algorithms on the unified feature space. Evaluate cluster purity relative to bug IDs or components.

\paragraph{Step 4: Perform bug prediction.}
Train and evaluate supervised classifiers.

\paragraph{Step 5: Text analysis.}
Apply TF–IDF or simple Transformers (optional) to bug summaries and alert notes. Extract key phrases and categories.

\paragraph{Step 6: Evaluate RCA quality.}
Measure:
\begin{itemize}
    \item cluster homogeneity,
    \item cluster completeness,
    \item classification accuracy for bug prediction,
    \item text-topic coherence.
\end{itemize}

\subsection{Experimental Design}

\paragraph{Experiment E1: Metadata + time-series clustering.}
Evaluate whether alerts with the same root cause form coherent clusters.

\paragraph{Experiment E2: Predictive models for bug filing.}
Train classifiers to predict which alerts lead to reported bugs.

\paragraph{Experiment E3: Text-derived root-cause categories.}
Extract topics and keywords from bug summaries. Compare with known components.

\paragraph{Experiment E4: Downstream vs.\ primary alert analysis.}
Determine whether downstream alerts can be connected back to their primary cause.

\subsection{Concrete Examples}

\paragraph{Example 1: Memory-related regressions.}
Bug summaries mention ``memory usage'' or ``leak''. Alerts cluster based on increased variance.

\paragraph{Example 2: Graphics regressions.}
Alerts from rendering or layout tests may cluster together.

\paragraph{Example 3: Regression chains.}
Downstream alerts share metadata and time-series features with the primary alert.

\subsection{Expected Outcomes}
\begin{itemize}
    \item A multi-task RCA module integrating clustering, prediction, and text analysis.
    \item Evidence on whether automated RCA is feasible using Mozilla’s dataset.
    \item A list of dominant regression categories extracted from bug data.
    \item Insights into alert linkages and relationships.
\end{itemize}

\subsection{Reproducibility and Reporting}
The student will produce:
\begin{itemize}
    \item unified feature matrices,
    \item clustering scripts,
    \item bug-prediction experiments,
    \item text-analysis tools,
    \item a comprehensive RCA report.
\end{itemize}

\subsection{Risks and Mitigations}
\begin{itemize}
    \item \textbf{Sparse textual data:} bug summaries may be short; simple TF–IDF works better than deep NLP.
    \item \textbf{Label ambiguity:} some alerts may not map cleanly to a single root cause.
    \item \textbf{High-dimensional feature space:} apply PCA or UMAP before clustering.
\end{itemize}

\subsection{Phase 6 Deliverables}
\begin{itemize}
    \item Automated RCA tools (clustering, bug prediction, text analysis).
    \item Evaluation of RCA accuracy and usability.
    \item Case studies demonstrating successful RCA predictions.
    \item A final report integrating insights from all previous phases.
\end{itemize}



\end{document}
